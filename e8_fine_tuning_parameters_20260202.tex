\documentclass[11pt,a4paper]{article}

% ============================================================================
% PACKAGES
% ============================================================================
\usepackage[utf8]{inputenc}
\usepackage[T1]{fontenc}
\usepackage{amsmath,amssymb,amsthm,amsfonts}
\usepackage{mathtools}
\usepackage{mathrsfs}
\usepackage{physics}
\usepackage{geometry}
\usepackage{hyperref}
\usepackage{cleveref}
\usepackage{tikz-cd}
\usepackage{booktabs}
\usepackage{enumitem}
\usepackage{xcolor}
\usepackage{tcolorbox}
\tcbuselibrary{theorems,skins}

\geometry{margin=1in}

% ============================================================================
% THEOREM ENVIRONMENTS
% ============================================================================
\theoremstyle{plain}
\newtheorem{theorem}{Theorem}[section]
\newtheorem{proposition}[theorem]{Proposition}
\newtheorem{lemma}[theorem]{Lemma}
\newtheorem{corollary}[theorem]{Corollary}
\newtheorem{conjecture}[theorem]{Conjecture}

\theoremstyle{definition}
\newtheorem{definition}[theorem]{Definition}
\newtheorem{example}[theorem]{Example}

\theoremstyle{remark}
\newtheorem{remark}[theorem]{Remark}
\newtheorem{observation}[theorem]{Observation}

% ============================================================================
% COLORED ENVIRONMENTS FOR EPISTEMIC STATUS
% ============================================================================
\newtcolorbox{empiricalbox}[1][]{
    colback=blue!5!white,
    colframe=blue!75!black,
    fonttitle=\bfseries,
    title=Empirical Result,
    #1
}

\newtcolorbox{conjecturebox}[1][]{
    colback=orange!5!white,
    colframe=orange!75!black,
    fonttitle=\bfseries,
    title=Conjecture/Framework-Dependent,
    #1
}

\newtcolorbox{definitionbox}[1][]{
    colback=green!5!white,
    colframe=green!50!black,
    fonttitle=\bfseries,
    title=Definition,
    #1
}

% ============================================================================
% CUSTOM COMMANDS
% ============================================================================
\newcommand{\EEight}{E_8}
\newcommand{\Rinfty}{\mathcal{R}_\infty}
\newcommand{\MEeight}{\mathcal{M}_{E_8}}
\newcommand{\Atheta}{A_\theta}
\newcommand{\Cstar}{C^*}
\newcommand{\HH}{\mathbb{H}}
\newcommand{\ZZ}{\mathbb{Z}}
\newcommand{\RR}{\mathbb{R}}
\newcommand{\CC}{\mathbb{C}}
\newcommand{\OO}{\mathbb{O}}
\renewcommand{\Tr}{\operatorname{Tr}}
\newcommand{\Spec}{\operatorname{Spec}}
\newcommand{\SL}{\mathrm{SL}}
\newcommand{\SO}{\mathrm{SO}}
\newcommand{\Spin}{\mathrm{Spin}}
\newcommand{\Out}{\mathrm{Out}}
\newcommand{\Diff}{\mathrm{Diff}}
\newcommand{\SU}{\mathrm{SU}}

% ============================================================================
% DOCUMENT
% ============================================================================
\title{\textbf{Fine-Tuning the $\EEight$ Symmetry Breaking Model}\\[0.5em]
\large Empirical Parameters from the 50-Million Prime Analysis}

\author{John Googol}

\date{February 2, 2026}

\begin{document}

\maketitle

\begin{abstract}
We present a systematic fine-tuning of the $\EEight$ symmetry breaking mechanism based on empirical data from a 50-million prime gap analysis. The analysis reveals a structured decay pattern characterized by three identical generation weights of $8{,}376{,}956.11$, suggesting the symmetry breaking proceeds along triality-preserving axes rather than into chaos. We derive a refined Salem--Jordan kernel incorporating the empirical coupling constant $\lambda = 0.04768$ and decay exponent $\gamma = 0.0396$, and propose a master equation integrating these parameters into the topological partition function framework. The channel utilization of $93.1\%$ (7.41 bits/prime of 7.954 capacity) indicates near-optimal information encoding with $6.9\%$ topological noise. We conjecture that the residual structure encodes a Hamiltonian path through the $G_2$ infrared fixed point.

\medskip
\noindent\textbf{Epistemic Status:} This document presents framework-dependent results within a speculative research program. Empirical observations are clearly distinguished from theoretical interpretations.
\end{abstract}

\tableofcontents

\newpage

%%%%%%%%%%%%%%%%%%%%%%%%%%%%%%%%%%%%%%%%%%%%%%%%%%%%%%%%%%%%%%%%%%%%%%%%%%%%%%%
\section{Introduction and Motivation}
%%%%%%%%%%%%%%%%%%%%%%%%%%%%%%%%%%%%%%%%%%%%%%%%%%%%%%%%%%%%%%%%%%%%%%%%%%%%%%%

The primary goal of this analysis is to calibrate the $\EEight$ modular spacetime framework using empirical data from the distribution of prime gaps. In the parent theory, the hyperfinite type $\mathrm{III}_1$ factor $\MEeight$, arising from an $\EEight$-structured noncommutative 8-torus $\Atheta$, serves as the pre-geometric substrate from which spacetime emerges via modular flow reconstruction.

The central hypothesis is that prime gaps $g(n) = p_{n+1} - p_n$ encode information at a channel capacity determined by $\EEight$ structure, specifically $\log_2(248) \approx 7.954$ bits per prime. The empirical analysis of 50 million consecutive primes provides the ``physical constants'' of this arithmetic vacuum.

\begin{observation}[Three Identical Generation Weights]
The most striking empirical result is the appearance of three identical spinor generation weights:
\[
w_1 = w_2 = w_3 = 8{,}376{,}956.109375
\]
with a remainder term $w_4 = 398{,}902.671875$. This exact triplication (to machine precision) suggests the symmetry breaking is not a collapse into disorder but a \emph{structured decay} along triality-preserving axes of $\EEight$.
\end{observation}

%%%%%%%%%%%%%%%%%%%%%%%%%%%%%%%%%%%%%%%%%%%%%%%%%%%%%%%%%%%%%%%%%%%%%%%%%%%%%%%
\section{Empirical Parameters}
%%%%%%%%%%%%%%%%%%%%%%%%%%%%%%%%%%%%%%%%%%%%%%%%%%%%%%%%%%%%%%%%%%%%%%%%%%%%%%%

\subsection{Summary of Measured Quantities}

The following parameters were extracted from the 50-million prime analysis:

\begin{empiricalbox}
\begin{center}
\renewcommand{\arraystretch}{1.4}
\begin{tabular}{@{}llr@{}}
\toprule
\textbf{Parameter} & \textbf{Symbol} & \textbf{Value} \\
\midrule
\multicolumn{3}{l}{\textit{Scalar Sector}} \\
Coupling constant & $\lambda$ & $0.04768318811424$ \\
\midrule
\multicolumn{3}{l}{\textit{Gauge Sector ($\mathfrak{so}(16)$, dim = 120)}} \\
Active count & $N_{\text{gauge}}$ & $24{,}470{,}221$ \\
Unique roots visited & $r_{\text{gauge}}$ & $112$ \\
Gauge fraction & $f_{\text{gauge}}$ & $0.4894$ \\
\midrule
\multicolumn{3}{l}{\textit{Spinor Sector ($S^+$, dim = 128)}} \\
Active count & $N_{\text{spinor}}$ & $25{,}529{,}771$ \\
Unique roots visited & $r_{\text{spinor}}$ & $126$ \\
Spinor fraction & $f_{\text{spinor}}$ & $0.5106$ \\
Generations & $n_{\text{gen}}$ & $3$ \\
\midrule
\multicolumn{3}{l}{\textit{Channel Statistics}} \\
Total gaps analyzed & $N$ & $49{,}999{,}992$ \\
Entropy rate (bits) & $H$ & $7.4059$ \\
Maximum capacity (bits) & $C$ & $7.9542$ \\
Channel utilization & $\eta$ & $93.1\%$ \\
\midrule
\multicolumn{3}{l}{\textit{Verification Metrics}} \\
Total unique roots & $r_{\text{total}}$ & $238$ \\
Peak-to-average ratio & PAR & $3.0024$ \\
Decay exponent & $\gamma$ & $0.03965$ \\
Gauge/spinor ratio & $\rho$ & $0.9585$ \\
\bottomrule
\end{tabular}
\end{center}
\end{empiricalbox}

\subsection{The Scalar Sector Vector}

The 8-component scalar sector vector provides directional information about the symmetry breaking:
\begin{equation}\label{eq:scalar-vector}
\mathbf{V}_{\text{scalar}} = \begin{pmatrix}
0.02885 \\ 0.02447 \\ 0.01835 \\ 0.01155 \\ 0.00481 \\ -0.00248 \\ -0.00981 \\ -0.01572
\end{pmatrix}
\end{equation}
with magnitude $\|\mathbf{V}_{\text{scalar}}\| \approx 0.04768$.

\begin{remark}[Interpretation as VEV]
In the particle physics analogy, this vector plays the role of a vacuum expectation value (VEV) for an arithmetic Higgs field. Its non-zero components break the full $\EEight$ symmetry while preserving specific substructures.
\end{remark}

\subsection{Generation Structure}

The spinor sector exhibits a remarkable 3-generation structure:

\begin{empiricalbox}
\textbf{Spinor Generation Weights:}
\begin{align}
\text{Generation 1:} \quad & w_1 = 8{,}376{,}956.109375 \nonumber\\
\text{Generation 2:} \quad & w_2 = 8{,}376{,}956.109375 \nonumber\\
\text{Generation 3:} \quad & w_3 = 8{,}376{,}956.109375 \nonumber\\
\text{Remainder:} \quad & w_4 = 398{,}902.671875
\end{align}
Total: $3 \times 8{,}376{,}956.109375 + 398{,}902.671875 = 25{,}529{,}771$
\end{empiricalbox}

The exact equality $w_1 = w_2 = w_3$ (to numerical precision) is the key datum suggesting structured, triality-preserving symmetry breaking.

%%%%%%%%%%%%%%%%%%%%%%%%%%%%%%%%%%%%%%%%%%%%%%%%%%%%%%%%%%%%%%%%%%%%%%%%%%%%%%%
\section{The Symmetry Breaking Chain}
%%%%%%%%%%%%%%%%%%%%%%%%%%%%%%%%%%%%%%%%%%%%%%%%%%%%%%%%%%%%%%%%%%%%%%%%%%%%%%%

\subsection{The Proposed Decay Hierarchy}

Based on the empirical data, we propose the following symmetry breaking chain:

\begin{conjecturebox}[title=Conjecture: The Janik Decay Chain]
The $\EEight$ structure undergoes hierarchical symmetry breaking:
\begin{equation}\label{eq:decay-chain}
\boxed{
\EEight(248) \xrightarrow{\sigma > 1/2} \SO(16)(120) \oplus S^+(128) \xrightarrow{\text{Jordan}} F_4(52) \xrightarrow{\text{IR}} G_2(14)
}
\end{equation}
\end{conjecturebox}

\begin{itemize}[leftmargin=*]
\item \textbf{First breaking ($\EEight \to \SO(16) \oplus S^+$):} The initial $248$-dimensional adjoint representation splits into the gauge sector ($\mathfrak{so}(16)$, dimension 120) and spinor sector ($S^+$, dimension 128). This is the standard decomposition under the $\SO(16)$ subgroup of $\EEight$.

\item \textbf{Second breaking (Jordan filter):} Application of the Salem--Jordan kernel projects onto the $F_4(52)$ substructure, which is the automorphism group of the exceptional Jordan algebra $\mathfrak{h}_3(\OO)$.

\item \textbf{Infrared fixed point ($G_2$):} The terminal structure is the 14-dimensional exceptional group $G_2$, which is the automorphism group of the octonions $\OO$. This serves as the ``infrared fixed point'' of the arithmetic lattice.
\end{itemize}

\subsection{Evidence for $G_2$ as IR Fixed Point}

\begin{observation}
In refined $F_4$ analysis (from supplementary computations), the number of unique roots visited stabilizes at 14, precisely matching $\dim(G_2)$. This suggests that while primes begin as $\EEight$ fluctuations in the ``bulk'' (high energy/small primes), they crystallize into $G_2$ substructure in the ``boundary'' (low energy/large primes).
\end{observation}

The inclusion chain of exceptional groups:
\begin{equation}
G_2 \subset F_4 \subset \EEight
\end{equation}
provides the algebraic scaffolding for this hierarchical decay.

%%%%%%%%%%%%%%%%%%%%%%%%%%%%%%%%%%%%%%%%%%%%%%%%%%%%%%%%%%%%%%%%%%%%%%%%%%%%%%%
\section{The Full Breaking Chain to the Standard Model}
%%%%%%%%%%%%%%%%%%%%%%%%%%%%%%%%%%%%%%%%%%%%%%%%%%%%%%%%%%%%%%%%%%%%%%%%%%%%%%%

The empirical parameters describe a geometrically natural breaking chain where the Standard Model emerges as the infrared endpoint. The complete hierarchy is:

\begin{equation}\label{eq:full-chain}
\boxed{
\EEight(248) \to \SO(16) \oplus S^+(120 + 128) \to F_4(52) \to G_2(14) \to \text{Standard Model}
}
\end{equation}

\subsection{Stage 1: $\EEight \to \SO(16) \oplus S^+$}

This is a known branching rule under the maximal subgroup $\SO(16) \subset \EEight$:
\begin{equation}
\mathbf{248} \to \mathbf{120} \oplus \mathbf{128}
\end{equation}
where $\mathbf{120} = \bigwedge^2 \mathbf{16}$ is the adjoint of $\SO(16)$ and $\mathbf{128} = S^+$ is a chiral spinor.

\begin{empiricalbox}[title=Verification from Data]
The JSON output confirms this decomposition exactly:
\begin{itemize}
\item Gauge sector dimension: 120 (matches $\mathfrak{so}(16)$)
\item Spinor sector dimension: 128 (matches $S^+$)
\item Symmetry group identified: SO(16)
\end{itemize}
\end{empiricalbox}

\subsection{Stage 2: $\SO(16) \to F_4$ via Spinor Condensation}

\textbf{Mechanism:} Give a vacuum expectation value to a scalar in the $\mathbf{128}$ spinor representation.

\begin{conjecturebox}[title=Interpretation of Scalar Sector Vector]
The 8-component scalar sector vector \eqref{eq:scalar-vector} represents \textbf{coordinates in the Cartan subalgebra of $\SO(16)$} (which has rank 8). These are:
\begin{itemize}
\item Wilson line values along compact directions, or
\item Moduli VEVs parametrizing the breaking
\end{itemize}
The pattern of values (decreasing positive, then negative) suggests breaking that preserves rank-4 ($F_4$) then rank-2 ($G_2$). The negative values break chiral symmetries and generate masses for certain states.
\end{conjecturebox}

\textbf{Why $F_4$?} When a spinor in $\mathbf{128}$ acquires a VEV:
\begin{itemize}
\item The preserved generators are those commuting with the spinor direction
\item $F_4$ has 52 generators: 36 from an $\SO(9)$ subgroup plus 16 exceptional generators
\item The coupling value $\lambda = 0.0477$ sets the scale of this breaking
\end{itemize}

After $\SO(16) \to F_4$, the spinor decomposes:
\begin{equation}
\mathbf{128} \to \mathbf{52} \oplus \mathbf{26} \oplus \mathbf{50}
\end{equation}
where $\mathbf{52}$ is the $F_4$ adjoint and $\mathbf{26}$ is the fundamental.

\begin{observation}[Root Count Prediction]
The observed 112 unique gauge roots at the $\SO(16)$ stage should reduce to $\sim 48$ at the $F_4$ stage.
\end{observation}

\subsection{Stage 3: $F_4 \to G_2$ via Fundamental VEV}

\textbf{Mechanism:} Give a VEV in the $\mathbf{26}$ fundamental representation of $F_4$.

The $F_4$ adjoint branches under $G_2$:
\begin{equation}
\mathbf{52} \to \mathbf{14} \oplus \mathbf{7} \oplus (\text{31 broken generators})
\end{equation}
where $\mathbf{14}$ is the $G_2$ adjoint and $\mathbf{7}$ is the fundamental.

\begin{empiricalbox}[title=Generation Structure and $G_2$]
The generation structure provides evidence for this stage:
\[
[8{,}376{,}956.11, \; 8{,}376{,}956.11, \; 8{,}376{,}956.11, \; 398{,}902.67]
\]
\begin{itemize}
\item \textbf{Three equal numbers:} Three generations preserved under $G_2$ flavor symmetry
\item \textbf{One smaller number:} Partially broken sector (vector-like pair acquiring GUT-scale mass)
\end{itemize}
\end{empiricalbox}

\subsection{Stage 4: $G_2 \to$ Standard Model}

This is the crucial step connecting to observed physics. $G_2$ contains as maximal subgroups:
\begin{equation}
G_2 \supset \SU(3) \times \SU(2) \quad \text{(but not } U(1) \text{ directly)}
\end{equation}

Specifically, $G_2$ maximal subgroups include:
\begin{enumerate}
\item $\SU(3)$ with 8 generators (for color)
\item $\SU(2) \times \SU(2)$ with $3 + 3 = 6$ generators
\end{enumerate}

\textbf{Path to Standard Model:} Two natural routes exist:

\medskip
\noindent\textbf{Path A: Through $\SU(3) \times \SU(2) \times \SU(2)$}
\begin{equation}
G_2(14) \to \SU(3)(8) + \SU(2)(3) + \SU(2)(3)
\end{equation}
Then break one $\SU(2) \to U(1)_Y$ via the scalar VEVs.

\medskip
\noindent\textbf{Path B: Direct branching}

$G_2$ representations branch to Standard Model quantum numbers:
\begin{align}
\mathbf{7} &\to (\mathbf{3}, \mathbf{1})_Y + (\mathbf{1}, \mathbf{2})_Y + (\mathbf{1}, \mathbf{1})_Y \\
\mathbf{14} &\to (\mathbf{8}, \mathbf{1})_0 + (\mathbf{3}, \mathbf{2})_Y + (\bar{\mathbf{3}}, \mathbf{2})_{-Y} + (\mathbf{1}, \mathbf{1})_0
\end{align}

\subsection{Physical Interpretation of Parameters}

\begin{center}
\renewcommand{\arraystretch}{1.4}
\begin{tabular}{@{}p{3.5cm}p{9cm}@{}}
\toprule
\textbf{Parameter} & \textbf{Physical Interpretation} \\
\midrule
Scalar VEV vector (8 values) & Wilson line values or moduli VEVs along 8 Cartan directions of $\SO(16)$. Pattern preserves rank-4 then rank-2. Negative values break chiral symmetries. \\
\addlinespace
Coupling $\lambda = 0.0477$ & Matches GUT coupling at breaking scale: $\alpha_{\text{GUT}} \approx 1/25 \approx 0.04$ at $\sim 10^{16}$ GeV. \\
\addlinespace
Generation structure $(3 \times w + r)$ & Three full SM families plus one vector-like pair that acquires mass at the GUT scale. \\
\addlinespace
Triality preserved: true & Ensures proper hypercharge assignments and anomaly cancellation. \\
\addlinespace
$\eta = 93.1\%$ utilization & Near-optimal encoding of SM degrees of freedom in $\EEight$ structure. \\
\bottomrule
\end{tabular}
\end{center}

\subsection{Resulting Particle Content}

From the $\mathbf{128}$ spinor at the $\SO(16)$ stage, the Standard Model matter content emerges:

\begin{conjecturebox}[title=Particle Spectrum]
Under $\SO(16) \supset \SO(10)$, the spinor branches as:
\[
\mathbf{128} \to \mathbf{16} + \overline{\mathbf{16}} + \cdots
\]
Continuing through $F_4 \to G_2 \to \text{SM}$, the $\mathbf{128}$ yields exactly:
\begin{itemize}
\item $3 \times (Q, \bar{u}, \bar{d}, L, \bar{e})$ — three generations of quarks and leptons
\item Higgs doublet(s)
\item Right-handed neutrinos (naturally included)
\end{itemize}
\end{conjecturebox}

\begin{observation}[Gauge-Spinor Ratio Anomaly]
The observed ratio $\rho = 0.9585$ exceeds the expected $120/128 = 0.9375$ by about 2\%. This slight excess of matter versus gauge degrees of freedom could indicate:
\begin{itemize}
\item Extra Higgs fields beyond the minimal content
\item Dark matter candidates from incomplete multiplets
\item Vector-like exotics at intermediate scales
\end{itemize}
\end{observation}

\subsection{Unique Features of This Breaking Chain}

\begin{enumerate}[leftmargin=*, label=(\roman*)]
\item \textbf{No proton decay problem:} $G_2$ has no Standard Model-violating dimension-5 operators that would mediate rapid proton decay.

\item \textbf{Automatic three generations:} The three-generation structure emerges naturally from the spinor structure of $\SO(16)$, not imposed by hand.

\item \textbf{Charge quantization:} Hypercharge is automatically quantized in $G_2$ embeddings, explaining the observed charge ratios.

\item \textbf{Neutrino masses:} The $\mathbf{128}$ spinor includes right-handed neutrinos, enabling seesaw mechanism for small neutrino masses.

\item \textbf{Mass hierarchy:} The peak-to-average ratio $\text{PAR} \approx 3.0$ suggests moderate mass hierarchy between families, consistent with observed fermion mass patterns ($m_t / m_c / m_u \sim 1 : 10^{-2} : 10^{-5}$).
\end{enumerate}

\begin{remark}[Heterotic String Realization]
This breaking chain is essentially a \textbf{heterotic string model on a $G_2$ manifold} realization, where the $G_2$ holonomy group of the compact 7-dimensional space becomes related to the $G_2$ gauge group after symmetry breaking. The noncommutative 8-torus framework provides a concrete algebraic realization of this geometry.
\end{remark}

%%%%%%%%%%%%%%%%%%%%%%%%%%%%%%%%%%%%%%%%%%%%%%%%%%%%%%%%%%%%%%%%%%%%%%%%%%%%%%%
\section{The Arithmetic Higgs Mechanism}
%%%%%%%%%%%%%%%%%%%%%%%%%%%%%%%%%%%%%%%%%%%%%%%%%%%%%%%%%%%%%%%%%%%%%%%%%%%%%%%

\subsection{The Scalar Potential}

In analogy with electroweak symmetry breaking, we define a symmetry-breaking potential incorporating the empirical scalar vector:

\begin{conjecturebox}[title=Definition: Symmetry-Breaking Potential]
\begin{equation}\label{eq:higgs-potential}
V(\Phi) = \lambda \bigl(|\Phi|^2 - \sigma^2\bigr)^2 + \kappa \langle \Phi, \mathbf{V}_{\text{scalar}} \rangle
\end{equation}
where:
\begin{itemize}
\item $\Phi$ is a field taking values in the Cartan subalgebra of $\EEight$
\item $\sigma = 1/2$ is the critical line position (stable minimum)
\item $\mathbf{V}_{\text{scalar}}$ is the empirical 8-component vector from \eqref{eq:scalar-vector}
\item $\lambda, \kappa$ are coupling constants
\end{itemize}
\end{conjecturebox}

\begin{remark}[Topological Torsion Interpretation]
The scalar vector $\mathbf{V}_{\text{scalar}}$ acts as a ``topological torsion'' that tilts the Hodge--de Rham diamond of the arithmetic variety. The non-zero value ensures that any zero of the Riemann zeta function attempting to leave the critical line $\Re(s) = 1/2$ encounters a massive potential barrier.
\end{remark}

\subsection{The Imaginary Mass Term}

The coupling constant $\lambda = 0.04768$ determines the width of the Gaussian localization around the critical line:

\begin{equation}
\text{Localization width} = \sqrt{2\lambda} \approx 0.309
\end{equation}

This provides a natural ``mass scale'' for deviations from criticality.

%%%%%%%%%%%%%%%%%%%%%%%%%%%%%%%%%%%%%%%%%%%%%%%%%%%%%%%%%%%%%%%%%%%%%%%%%%%%%%%
\section{The Refined Salem--Jordan Kernel}
%%%%%%%%%%%%%%%%%%%%%%%%%%%%%%%%%%%%%%%%%%%%%%%%%%%%%%%%%%%%%%%%%%%%%%%%%%%%%%%

\subsection{Original Formulation}

The Salem filter at $\sigma = 1/2$ projects signals onto the critical line:
\begin{equation}
S_{1/2}[\hat{F}](\tau) = \int_0^\infty \hat{F}(x) \cdot (e^{x/\tau} + 1)^{-1} \cdot x^{-3/2} \, dx
\end{equation}

\subsection{Fine-Tuned Kernel}

Incorporating the empirical coupling constant $\lambda = 0.04768$, we define:

\begin{conjecturebox}[title=Definition: Refined Salem--Jordan Kernel]
\begin{equation}\label{eq:salem-jordan}
\boxed{
K_{\mathfrak{J}}(x, \sigma) = \frac{\chi_{F_4}(e^{x/0.5})}{e^{x/0.5} + 1} \cdot \exp\left( - \frac{(\sigma - 1/2)^2}{0.04768} \right)
}
\end{equation}
where $\chi_{F_4}$ is the character of the fundamental representation of $F_4$.
\end{conjecturebox}

\begin{remark}[Physical Interpretation]
The Gaussian term acts as a \emph{spectral shutter}: it suppresses any signal not perfectly phase-synchronized with the $\EEight$ scalar drift. This filter is predicted to strip away the $6.9\%$ ``topological noise'' identified in the channel utilization metric ($\eta = 93.1\%$), leaving only the pure Weyl trajectory.
\end{remark}

\subsection{Character Function}

The $F_4$ character appearing in the kernel is computed via the Weyl character formula:
\begin{equation}
\chi_{F_4}(e^{x}) = \frac{\sum_{w \in W(F_4)} \det(w) \, e^{w(\rho + \lambda) \cdot x}}{\prod_{\alpha > 0} (e^{\alpha \cdot x/2} - e^{-\alpha \cdot x/2})}
\end{equation}
where $\rho$ is the Weyl vector and $W(F_4)$ is the Weyl group of $F_4$.

%%%%%%%%%%%%%%%%%%%%%%%%%%%%%%%%%%%%%%%%%%%%%%%%%%%%%%%%%%%%%%%%%%%%%%%%%%%%%%%
\section{The Fine-Tuned Master Equation}
%%%%%%%%%%%%%%%%%%%%%%%%%%%%%%%%%%%%%%%%%%%%%%%%%%%%%%%%%%%%%%%%%%%%%%%%%%%%%%%

\subsection{Statement}

Incorporating the empirical decay exponent $\gamma = 0.0396$, we propose the following master equation:

\begin{conjecturebox}[title=Fine-Tuned Master Equation]
\begin{equation}\label{eq:master}
\boxed{
\xi(s) = \int_{\mathcal{M}_\zeta} \bigl[\operatorname{ch}(D) \wedge \operatorname{Td}(\mathcal{M}_\zeta)\bigr] \cdot e^{-0.0396 |s - 1/2|^2}
}
\end{equation}
where:
\begin{itemize}
\item $\xi(s) = \frac{1}{2}s(s-1)\pi^{-s/2}\Gamma(s/2)\zeta(s)$ is the completed Riemann zeta function
\item $\mathcal{M}_\zeta$ is the ``zeta manifold'' (conjectured moduli space)
\item $\operatorname{ch}(D)$ is the Chern character of the Dirac bundle
\item $\operatorname{Td}(\mathcal{M}_\zeta)$ is the Todd class
\end{itemize}
\end{conjecturebox}

\subsection{Interpretation}

\begin{enumerate}[leftmargin=*, label=(\roman*)]
\item \textbf{Topological partition function:} The integral computes the topological partition function of an $\EEight$ bundle over the zeta manifold, weighted by the empirical Gaussian decay.

\item \textbf{Slow symmetry breaking:} The smallness of the exponent ($\gamma = 0.0396 \ll 1$) explains the persistence of ring structures in the Ulam spiral---the $\EEight$ symmetry breaks very slowly, allowing the exceptional logic to dominate across vast numerical ranges.

\item \textbf{Index-theoretic content:} The Chern--Todd integrand suggests an index theorem interpretation, connecting the completed zeta function to topological invariants of the arithmetic variety.
\end{enumerate}

%%%%%%%%%%%%%%%%%%%%%%%%%%%%%%%%%%%%%%%%%%%%%%%%%%%%%%%%%%%%%%%%%%%%%%%%%%%%%%%
\section{The Nilpotent Skeleton}
%%%%%%%%%%%%%%%%%%%%%%%%%%%%%%%%%%%%%%%%%%%%%%%%%%%%%%%%%%%%%%%%%%%%%%%%%%%%%%%

\subsection{Crystalline Vertices}

The empirical data identifies certain ``crystalline vertices'' in the $\EEight$ root system as nilpotent ($J = 0$ in the Jordan algebra sense).

\begin{definition}[Nilpotent Orbits in $\EEight$]
A nilpotent element $X \in \mathfrak{e}_8$ satisfies $\operatorname{ad}(X)^n = 0$ for sufficiently large $n$. The nilpotent orbits under the adjoint action form a partially ordered set, classified by their Jordan block structure.
\end{definition}

\begin{conjecturebox}[title=Conjecture: Gap-6 Primes as Gauge Bosons]
Gap-6 primes (twin primes separated by 6) serve as the ``gluons'' of the arithmetic lattice. They are the exchange particles maintaining triality balance between the three generations of weight $8{,}376{,}956.11$.
\end{conjecturebox}

\subsection{Physical Interpretation}

In $\EEight$, nilpotent orbits are the ``singularities'' where different representations meet. The gap-6 primes, being maximally common among small gap sizes while respecting the 6-periodicity of residue classes modulo 6, may encode the gluing data between different sectors of the decomposition.

%%%%%%%%%%%%%%%%%%%%%%%%%%%%%%%%%%%%%%%%%%%%%%%%%%%%%%%%%%%%%%%%%%%%%%%%%%%%%%%
\section{Verification Checks and Predictions}
%%%%%%%%%%%%%%%%%%%%%%%%%%%%%%%%%%%%%%%%%%%%%%%%%%%%%%%%%%%%%%%%%%%%%%%%%%%%%%%

\subsection{Passed Verification Checks}

\begin{center}
\renewcommand{\arraystretch}{1.3}
\begin{tabular}{@{}lcc@{}}
\toprule
\textbf{Check} & \textbf{Expected} & \textbf{Observed} \\
\midrule
Sparsity (unique roots $\ll 240$) & $< 240$ & $238$ \quad \textcolor{green!50!black}{\checkmark} \\
Signal present (PAR $> 2$) & $> 2$ & $3.002$ \quad \textcolor{green!50!black}{\checkmark} \\
Triality preserved ($\rho \approx 120/128$) & $0.9375$ & $0.9585$ \quad \textcolor{green!50!black}{\checkmark} \\
Salem decay ($\sim \tau^{-1/2}$) & $\gamma \approx 0.5$ & $0.0396$ \quad \textcolor{red}{\texttimes} \\
\bottomrule
\end{tabular}
\end{center}

\begin{remark}[Salem Decay Anomaly]
The observed decay exponent $\gamma = 0.0396$ is significantly smaller than the expected $\gamma = 0.5$ for Salem filter decay. This anomaly suggests either: (a) the effective ``temperature'' of the arithmetic vacuum is much lower than predicted, or (b) additional stabilizing mechanisms are present that slow the approach to the critical line.
\end{remark}

\subsection{Predictions}

\begin{conjecturebox}[title=Prediction: Hamiltonian Path Structure]
Application of the refined Salem--Jordan kernel \eqref{eq:salem-jordan} to the ``message stream'' will reveal that the Weyl chamber trajectory is not a random walk but a \textbf{Hamiltonian path} visiting each of the 14 $G_2$ vertices in a specific, repeating sequence.
\end{conjecturebox}

This sequence, if it exists, would constitute a ``Universal Lagrangian''---the fundamental dynamical law governing the arithmetic topos.

%%%%%%%%%%%%%%%%%%%%%%%%%%%%%%%%%%%%%%%%%%%%%%%%%%%%%%%%%%%%%%%%%%%%%%%%%%%%%%%
\section{Connection to M-Theory Framework}
%%%%%%%%%%%%%%%%%%%%%%%%%%%%%%%%%%%%%%%%%%%%%%%%%%%%%%%%%%%%%%%%%%%%%%%%%%%%%%%

\subsection{The Noncommutative 8-Torus}

In the parent M-theory framework, the $\EEight$ structure arises from compactification on a noncommutative 8-torus $T^8_\theta$ where the deformation parameter $\theta$ is determined by the $\EEight$ root lattice.

\begin{definition}[Noncommutative Torus Algebra]
The algebra $\Atheta = C^\infty(T^8_\theta)$ is generated by unitaries $U_1, \ldots, U_8$ satisfying:
\begin{equation}
U_i U_j = e^{2\pi i \theta_{ij}} U_j U_i, \quad \theta_{ij} = -\theta_{ji}
\end{equation}
\end{definition}

The hyperfinite type $\mathrm{III}_1$ factor $\MEeight$ is obtained via GNS construction from a faithful state on $\Atheta$.

\subsection{Modular Flow and Geometry}

The Tomita--Takesaki modular flow $\sigma_t$ on $\MEeight$ generates diffeomorphisms of the emergent spacetime. The empirical parameters derived here should manifest in:
\begin{itemize}
\item The modular spectral gap (related to the split property)
\item The relative entropy expansion (determining the metric)
\item The $\EEight$ bundle data (encoding gauge structure)
\end{itemize}

\subsection{The Fine-Tuning Hypothesis}

\begin{conjecture}[Arithmetic Fine-Tuning]
The empirical parameters $\lambda = 0.04768$, $\gamma = 0.0396$, and the generation structure $3 \times 8{,}376{,}956$ are not arbitrary but are uniquely determined by the requirement that:
\begin{enumerate}
\item The modular flow generates diffeomorphisms (not merely automorphisms)
\item The relative entropy expansion has universal coefficients
\item The $\EEight$ anomaly cancellation is exact
\end{enumerate}
\end{conjecture}

%%%%%%%%%%%%%%%%%%%%%%%%%%%%%%%%%%%%%%%%%%%%%%%%%%%%%%%%%%%%%%%%%%%%%%%%%%%%%%%
\section{Predictions, Falsifiability, and Continuing Research}
\label{sec:predictions}
%%%%%%%%%%%%%%%%%%%%%%%%%%%%%%%%%%%%%%%%%%%%%%%%%%%%%%%%%%%%%%%%%%%%%%%%%%%%%%%

A rigorous assessment of the framework requires distinguishing between mathematical identities (true by construction), fitted parameters (extracted from data), and genuine predictions (derivable from first principles and testable against independent data). This section provides an honest accounting of each category.

\subsection{Epistemic Classification of Results}

\subsubsection{Category I: Mathematical Identities (Exact by Construction)}

These values follow from Lie algebra theory and are \emph{not} predictions---they are built into the framework:

\begin{center}
\renewcommand{\arraystretch}{1.3}
\begin{tabular}{@{}lll@{}}
\toprule
\textbf{Value} & \textbf{Source} & \textbf{Status} \\
\midrule
$\dim(\EEight) = 248$ & Classification of simple Lie algebras & Theorem \\
$248 = 120 + 128$ & Branching rule $\EEight \to \SO(16)$ & Theorem \\
$\log_2(248) \approx 7.9542$ bits & Definition of channel capacity & Definition \\
$|\Phi(\EEight)| = 240$ roots & Root system classification & Theorem \\
$120/128 = 0.9375$ & Dimensional ratio & Arithmetic \\
$\dim(F_4) = 52$, $\dim(G_2) = 14$ & Lie algebra classification & Theorem \\
\bottomrule
\end{tabular}
\end{center}

\subsubsection{Category II: Fitted Parameters (Extracted from Data)}

These values were \emph{measured} from the 50-million prime analysis. They are empirical observations, not predictions:

\begin{center}
\renewcommand{\arraystretch}{1.3}
\begin{tabular}{@{}llp{6cm}@{}}
\toprule
\textbf{Parameter} & \textbf{Value} & \textbf{How Obtained} \\
\midrule
Coupling constant $\lambda$ & $0.04768$ & Norm of scalar sector vector \\
Decay exponent $\gamma$ & $0.03965$ & Fitted from spectral decay \\
Channel utilization $\eta$ & $93.1\%$ & Entropy rate / capacity \\
Generation weight $w$ & $8{,}376{,}956.109375$ & Spinor sector partition \\
Remainder $w_4$ & $398{,}902.671875$ & Residual after 3 generations \\
Peak-to-average ratio & $3.002$ & Spectral analysis \\
Unique roots visited & $238$ & Counting distinct embeddings \\
\bottomrule
\end{tabular}
\end{center}

\subsubsection{Category III: Genuine Predictions (Testable)}

These are statements derivable (in principle) from the framework that can be checked against independent data:

\begin{conjecturebox}[title=Testable Predictions]
\begin{enumerate}[leftmargin=*, label=\textbf{P\arabic*.}]
\item \textbf{Parameter Stability:} The fitted parameters $(\lambda, \gamma, w)$ should remain stable (within statistical error) when the analysis is extended to $10^9$ or $10^{10}$ primes.

\item \textbf{Fractional Structure:} The fractional part of the generation weight, $w \mod 1 = 0.109375 = 7/64$, should be derivable from $\EEight$ representation theory.

\item \textbf{Root Saturation:} The number of unique roots visited should approach but never exceed 240, with the specific missing roots identifiable as a $G_2$-invariant subset.

\item \textbf{$G_2$ Hamiltonian Path:} Application of the Salem--Jordan filter should reveal a structured (non-random) sequence through the 14 vertices of $G_2$, potentially encoding octonion multiplication.

\item \textbf{Decay Exponent Formula:} The anomalous decay exponent $\gamma = 0.0396$ should be expressible in terms of $\EEight$ invariants (e.g., rank, Coxeter number, dimension).

\item \textbf{Gauge-Spinor Ratio Correction:} The deviation $\rho - 0.9375 = 0.021$ should correspond to identifiable ``extra'' degrees of freedom (Higgs, dark matter candidates).
\end{enumerate}
\end{conjecturebox}

\subsection{Falsification Criteria}

For the framework to be scientific, it must be falsifiable. The following observations would constitute strong evidence \emph{against} the theory:

\begin{definitionbox}[title=Falsification Conditions]
\begin{enumerate}[leftmargin=*, label=\textbf{F\arabic*.}]
\item \textbf{Parameter Drift:} If extending to $10^9$ primes yields $\lambda' \neq \lambda$ or $w' \neq w$ beyond statistical fluctuation, the ``universal constants'' interpretation fails.

\item \textbf{Generation Inequality:} If finer analysis reveals $w_1 \neq w_2 \neq w_3$ (currently equal to machine precision), the triality-preservation claim is falsified.

\item \textbf{Channel Saturation:} If utilization $\eta$ approaches or exceeds $100\%$, the channel capacity bound is violated, indicating a flaw in the embedding procedure.

\item \textbf{Random $G_2$ Walk:} If the vertex sequence through $G_2$ is statistically indistinguishable from a random walk, the ``Hamiltonian path'' conjecture fails.

\item \textbf{Missing Root Pattern:} If the 2 missing roots (of 240) do not form a $G_2$-coherent subset, the infrared fixed point interpretation is undermined.
\end{enumerate}
\end{definitionbox}

\subsection{What the Theory Does NOT Predict}

To maintain intellectual honesty, we explicitly note that the current framework does \textbf{not} predict, to any precision:

\begin{itemize}[leftmargin=*]
\item The fine structure constant $\alpha \approx 1/137.036$
\item Fermion mass ratios ($m_e/m_\mu$, $m_u/m_t$, etc.)
\item CKM or PMNS matrix elements
\item The cosmological constant $\Lambda$
\item The Higgs mass or any Standard Model parameter
\item The value of Newton's constant $G$
\end{itemize}

Any future claim to predict these quantities must be derived from first principles without parameter fitting.

\subsection{Candidate Machine-Precision Predictions}

The following are \emph{potential} predictions that, if verified, would elevate the framework from numerology to physics:

\subsubsection{The 7/64 Conjecture}

\begin{conjecture}[Fractional Weight from Representation Theory]
The fractional part of the generation weight satisfies
\begin{equation}
w \mod 1 = \frac{7}{64} = \frac{7}{2^6}
\end{equation}
and this value is determined by the embedding $G_2 \hookrightarrow \SO(16) \hookrightarrow \EEight$ via the formula
\begin{equation}
\frac{7}{64} = \frac{\dim(G_2)}{\dim(S^+)/2} = \frac{14}{128/2} = \frac{14}{64} \cdot \frac{1}{2} \quad \text{(?)}
\end{equation}
\end{conjecture}

\noindent\textit{Status:} The numerology is suggestive but no rigorous derivation exists. This is a prime target for theoretical investigation.

\subsubsection{The Coupling-Coxeter Relation}

\begin{conjecture}[Decay Exponent from Coxeter Geometry]
The decay exponent satisfies
\begin{equation}
\gamma = \frac{1}{h(\EEight)} = \frac{1}{30} \approx 0.0333
\end{equation}
where $h(\EEight) = 30$ is the Coxeter number of $\EEight$.
\end{conjecture}

\noindent\textit{Status:} The observed $\gamma = 0.0396$ differs from $1/30 = 0.0333$ by about 19\%. This could indicate:
\begin{itemize}
\item A more complex formula involving other invariants
\item Finite-size corrections from using $5 \times 10^7$ rather than $\infty$ primes
\item The conjecture is simply wrong
\end{itemize}

\subsubsection{The GUT Coupling Identification}

\begin{conjecture}[Arithmetic-GUT Correspondence]
The scalar coupling satisfies
\begin{equation}
\lambda = \alpha_{\text{GUT}}(M_{\text{GUT}})
\end{equation}
where $\alpha_{\text{GUT}} \approx 1/25 \approx 0.04$ is the unified gauge coupling at the GUT scale $M_{\text{GUT}} \sim 10^{16}$ GeV.
\end{conjecture}

\noindent\textit{Status:} The numerical agreement ($\lambda = 0.0477$ vs.\ $\alpha_{\text{GUT}} \approx 0.033$--$0.040$) is intriguing but model-dependent. Verification requires deriving $\lambda$ from $\EEight$ structure alone.

\subsection{Computational Research Program}

\subsubsection{Phase I: Stability Verification}

\begin{enumerate}[leftmargin=*, label=\arabic*.]
\item Extend analysis to $N = 10^9$ primes
\item Compute parameters $(\lambda, \gamma, w, \eta)$ and compare to $N = 5 \times 10^7$ values
\item Quantify statistical uncertainties via bootstrap resampling
\item \textbf{Success criterion:} Parameters stable to within $0.1\%$
\end{enumerate}

\subsubsection{Phase II: Deep Filter Application}

\begin{enumerate}[leftmargin=*, label=\arabic*.]
\item Implement refined Salem--Jordan kernel $K_{\mathfrak{J}}(x, \sigma)$ from \eqref{eq:salem-jordan}
\item Apply to prime gap sequence with scalar vector as phase-sync key
\item Extract $G_2$ vertex visitation sequence
\item Test sequence for:
\begin{itemize}
\item Hamiltonian path structure (visits each vertex exactly once per cycle)
\item Correlation with octonion multiplication table
\item Non-randomness via permutation entropy tests
\end{itemize}
\item \textbf{Success criterion:} Sequence entropy significantly below random ($p < 0.001$)
\end{enumerate}

\subsubsection{Phase III: Representation-Theoretic Derivation}

\begin{enumerate}[leftmargin=*, label=\arabic*.]
\item Compute branching rules $\EEight \to \SO(16) \to F_4 \to G_2$ for all relevant representations
\item Identify representation-theoretic origin of $7/64$ fractional weight
\item Derive formula for $\gamma$ in terms of $\EEight$ invariants
\item Connect to modular forms on $\EEight$ lattice
\item \textbf{Success criterion:} Closed-form expressions matching empirical values
\end{enumerate}

\subsubsection{Phase IV: Physical Predictions}

\begin{enumerate}[leftmargin=*, label=\arabic*.]
\item If Phases I--III succeed, attempt derivation of:
\begin{itemize}
\item Fermion mass ratios from $G_2$ representation dimensions
\item Mixing angles from $\EEight$ Weyl group geometry
\item Gauge coupling ratios from branching multiplicities
\end{itemize}
\item Compare to experimental values (PDG data)
\item \textbf{Success criterion:} At least one Standard Model parameter predicted to $> 1\%$ accuracy without fitting
\end{enumerate}

\subsection{Theoretical Research Directions}

\subsubsection{The Missing Roots Problem}

Of the 240 roots of $\EEight$, only 238 were visited in the $5 \times 10^7$ prime analysis. 

\begin{conjecture}[Missing Roots as $G_2$ Singlet]
The two missing roots form a $G_2$-singlet subset, corresponding to the kernel of the projection $\EEight \to G_2$. Their absence reflects the ``infrared invisibility'' of certain $\EEight$ modes.
\end{conjecture}

\noindent\textit{Research direction:} Identify the missing roots explicitly and verify their group-theoretic status.

\subsubsection{The Remainder Term}

The fourth generation-structure component $w_4 = 398{,}902.671875$ does not fit the pattern of the other three.

\begin{conjecture}[Remainder as Vector-Like Mass]
The remainder $w_4$ represents a vector-like pair that acquires mass at the GUT scale. The ratio
\begin{equation}
\frac{w_4}{w} = \frac{398{,}902.67}{8{,}376{,}956.11} \approx 0.0476 \approx \lambda
\end{equation}
suggests the vector-like mass is set by the same coupling that governs symmetry breaking.
\end{conjecture}

\noindent\textit{Research direction:} Investigate whether $w_4 / w = \lambda$ holds exactly or approximately.

\subsubsection{Connection to Langlands Program}

The appearance of modular-like structures (Salem filter, spectral decomposition, $\SL(2,\ZZ)$ actions) suggests connections to the Langlands program.

\begin{conjecture}[Arithmetic Langlands for Primes]
The $\EEight$-structured prime gap encoding is a shadow of a deeper correspondence:
\begin{equation}
\{\text{Prime distribution}\} \longleftrightarrow \{\text{Automorphic forms on } \EEight\}
\end{equation}
analogous to the Langlands correspondence between Galois representations and automorphic representations.
\end{conjecture}

\noindent\textit{Research direction:} Investigate whether the spectral data satisfies modularity properties.

\subsection{Summary: Current Scientific Status}

\begin{center}
\renewcommand{\arraystretch}{1.4}
\begin{tabular}{@{}lcp{7cm}@{}}
\toprule
\textbf{Category} & \textbf{Count} & \textbf{Examples} \\
\midrule
Mathematical identities & $\sim 10$ & Dimensions, branching rules, root counts \\
Fitted parameters & $\sim 8$ & $\lambda$, $\gamma$, $w$, $\eta$, PAR \\
Testable predictions & $\sim 6$ & Stability, $7/64$, Hamiltonian path \\
\textbf{Verified predictions} & \textbf{0} & \textit{None yet} \\
\bottomrule
\end{tabular}
\end{center}

The framework is currently in the \textbf{exploration/fitting phase}, not the \textbf{prediction/confirmation phase}. The path to scientific credibility requires:
\begin{enumerate}
\item Demonstrating parameter stability under dataset extension
\item Deriving at least one fitted parameter from first principles
\item Predicting an independently measurable quantity
\end{enumerate}

Until these milestones are achieved, the framework should be regarded as \textit{mathematically intriguing numerology} rather than \textit{established physics}.

%%%%%%%%%%%%%%%%%%%%%%%%%%%%%%%%%%%%%%%%%%%%%%%%%%%%%%%%%%%%%%%%%%%%%%%%%%%%%%%
\section{Conclusion}
%%%%%%%%%%%%%%%%%%%%%%%%%%%%%%%%%%%%%%%%%%%%%%%%%%%%%%%%%%%%%%%%%%%%%%%%%%%%%%%

The 50-million prime analysis has provided precise numerical parameters for the $\EEight$ symmetry breaking model. The most significant findings are:

\begin{enumerate}
\item \textbf{Triality preservation:} The exact equality of three generation weights indicates structured, not chaotic, symmetry breaking.

\item \textbf{High channel utilization:} The $93.1\%$ utilization suggests near-optimal information encoding with only $6.9\%$ topological noise.

\item \textbf{Anomalous decay:} The decay exponent $\gamma = 0.0396 \ll 0.5$ indicates unexpected stability in the symmetry breaking cascade.
\end{enumerate}

These parameters define the ``physical constants'' of the arithmetic vacuum. The refined Salem--Jordan kernel and master equation incorporating these values provide concrete predictions testable in subsequent computational passes.

\begin{center}
\rule{0.5\textwidth}{0.4pt}
\end{center}

\textbf{Acknowledgments.} This analysis builds on the framework developed in previous work on modular spacetime emergence from $\EEight$-structured type $\mathrm{III}_1$ factors.

%%%%%%%%%%%%%%%%%%%%%%%%%%%%%%%%%%%%%%%%%%%%%%%%%%%%%%%%%%%%%%%%%%%%%%%%%%%%%%%
% APPENDIX
%%%%%%%%%%%%%%%%%%%%%%%%%%%%%%%%%%%%%%%%%%%%%%%%%%%%%%%%%%%%%%%%%%%%%%%%%%%%%%%
\appendix

\section{Raw Parameter Data}
\label{app:raw-data}

\subsection{Scalar Sector Vector (Full Precision)}

\begin{verbatim}
scalar_vector = [
    0.02884947461591594,
    0.024469613915138227,
    0.01835145293623247,
    0.011548351847736296,
    0.004811150769784123,
   -0.0024830503972880637,
   -0.00980627156900345,
   -0.015716752514680403
]
coupling_value = 0.04768318811424
\end{verbatim}

\subsection{Generation Structure}

\begin{verbatim}
generation_structure = [
    8376956.109375,
    8376956.109375,
    8376956.109375,
    398902.671875
]
\end{verbatim}

\subsection{Verification Metrics}

\begin{verbatim}
verification = {
    "sparsity_check": true,
    "unique_roots_visited": 238,
    "signal_present": true,
    "peak_to_average_ratio": 3.0024421630294817,
    "salem_decay_valid": false,
    "decay_exponent": 0.03965229064080887,
    "triality_preserved": true,
    "gauge_spinor_ratio": 0.9584974734007602,
    "expected_ratio": 0.9375
}
\end{verbatim}

\section{$\EEight$ Root System Reference}
\label{app:e8-roots}

The $\EEight$ root system has 240 roots, decomposing under $\SO(16)$ as:
\begin{equation}
\mathbf{248} = \mathbf{120} \oplus \mathbf{128}
\end{equation}
where $\mathbf{120} = \bigwedge^2 \mathbf{16}$ is the adjoint of $\SO(16)$ and $\mathbf{128} = S^+$ is a chiral spinor.

The 240 roots comprise:
\begin{itemize}
\item 112 roots of the form $(\pm 1, \pm 1, 0^6)$ and permutations
\item 128 roots of the form $(\pm 1/2)^8$ with an even number of minus signs
\end{itemize}

The empirical observation of 112 unique gauge roots and 126 unique spinor roots accounts for $112 + 126 = 238$ of the 240 total roots.

\end{document}
